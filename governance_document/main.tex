\documentclass{article}
\usepackage{graphicx} % Required for inserting images
\usepackage{hyperref}

\title{Governance of the Python Accelerator Middle Layer Collaboration}
\date{version 1.0.0, Approved 11 April 2025}

\begin{document}

\maketitle



\section{Purpose}

The Python Accelerator Middle Layer (pyAML) collaboration is an open source community with the purpose to develop and maintain a joint technology platform for \textbf{design}, \textbf{commissioning} and \textbf{operation} of particle accelerators.

This document provides guidelines for how the collaboration should operate. It may be reviewed on the initiative of the steering committee.

\section{Principles}

The collaboration works according to the following guiding principles:

\begin{enumerate}
    \item \textbf{We have collaboration as a core-value.}
    
    We aim to develop a diverse community of people from accelerator laboratories around the world cultivating an environment of inclusivity, cooperation, and sharing.

    \item \textbf{We aim for user-centered design.}

    We will enable users of all levels to use the software and participate in the project by keeping barriers low.

    The software should follow the principles and practices of modern software engineering, be easy to install on the systems used by the community, easy to configure and easy to start to use.

    We should have a modular architecture which allows users to configure, use and develop algorithms and tools.

    \item \textbf{We value our data.}

    Configuration data should be separated from source code, easily updated and follow a common standard which can be extended if required.

    We should have a standard for how to save measurement and analysis data together with appropriate metadata.

\end{enumerate}


\section{Roles and Responsibilities}

\subsection{Users}

The users are accelerator physicists, control system engineers, and operators working at accelerator laboratories around the world.

The users are a fundamental part of the community and user participation is essential to ensure that the project satisfies their needs. Users are therefore asked to participate in the community as much as possible and the collaboration intends to provide resources such as training to make it easy for them to participate.

User contributions can include:

\begin{itemize}
    \item Spreading awareness of the project.
    \item Informing the maintainers of the user perspective.
    \item Providing feedback on the interfaces and functionality.
    \item Reporting issues.
\end{itemize}

\subsection{Contributors}
\label{contributors}
Contributors are community members who contribute in concrete ways to the project. Anyone can become a contributor and contributions can take many forms. There is no expectation of commitment and no selection process to become a contributor.

Contributions can include:

\begin{itemize}
    \item Software development, testing and CI/CD development.
    \item Supporting and training new users.
    \item Reporting bugs.
    \item Participate in discussions on issues and merge/pull requests.
    \item Identifying requirements.
    \item Providing graphics and web design.
    \item Assisting with project infrastructure.
    \item Writing documentation.
    \item Fixing bugs.
    \item Proposing new features.
    \item Improving existing features.
    \item Physics testing and benchmarking.
\end{itemize}

\subsection{Maintainers}

Maintainers are community members who are committed to the continued development and maintenance of the project. They have decision-making power regarding the software architecture, software design and code base.

Responsibilities of the maintainers include:
\begin{itemize}
    \item Making contributions (see Subsection \ref{contributors})
    \item Ensuring that contributions align with the scope and architecture of the project when merged.
    \item Ensuring functionality over time.
    \item Manage backward compatibility.
    \item Reviewing and approving merge/pull requests.
    \item Setting up and managing CI/CD workflows.
    \item Ensure appropriate documentation.
    \item Manage repository page.
\end{itemize}

Maintainers are assigned by the steering committee. A community member who wants to become a maintainer should send a request to the steering committee and a decision will be made by vote. A maintainer who wants to resign from the role should also send this request to the steering committee. In extreme circumstances, the steering committee has the power to remove maintainers.

\subsection{Steering committee}

The steering committee is responsible for strategic planning and coordination of the project. The steering committee represents the community and it is therefore important to have a diversity among the members. The members of the steering committee should reflect the technical expertise of the community and laboratories.

The steering committee is responsible for:
\begin{itemize}
    \item Decisions of strategic importance for the project.
    \item Handle funding opportunities and applications.
    \item Define working groups and delegate subprojects.    
    \item Set priorities, define time plans and keep track of progress.
    \item Arbitrate in case of disagreement between the maintainers.
    \item Project coordination
    \item Community-building
\end{itemize}

The steering committee is chosen on a renewable two-year mandate. To be eligible to candidate to the steering committee you need to come from a lab which commits resources to the project.

\end{document}
